% THIS IS SIGPROC-SP.TEX - VERSION 3.1
% WORKS WITH V3.2SP OF ACM_PROC_ARTICLE-SP.CLS
% APRIL 2009
%
% It is an example file showing how to use the 'acm_proc_article-sp.cls' V3.2SP
% LaTeX2e document class file for Conference Proceedings submissions.
% ----------------------------------------------------------------------------------------------------------------
% This .tex file (and associated .cls V3.2SP) *DOES NOT* produce:
%       1) The Permission Statement
%       2) The Conference (location) Info information
%       3) The Copyright Line with ACM data
%       4) Page numbering
% ---------------------------------------------------------------------------------------------------------------
% It is an example which *does* use the .bib file (from which the .bbl file
% is produced).
% REMEMBER HOWEVER: After having produced the .bbl file,
% and prior to final submission,
% you need to 'insert'  your .bbl file into your source .tex file so as to provide
% ONE 'self-contained' source file.
%
% Questions regarding SIGS should be sent to
% Adrienne Griscti ---> griscti@acm.org
%
% Questions/suggestions regarding the guidelines, .tex and .cls files, etc. to
% Gerald Murray ---> murray@hq.acm.org
%
% For tracking purposes - this is V3.1SP - APRIL 2009

\documentclass{acm_proc_article-sp}

\usepackage{caption}
\usepackage{soul}
\usepackage{color}
\usepackage{url}
\usepackage{hyperref}
\usepackage{subfig}
\usepackage{graphicx}
%\usepackage{algpseudocode}

\begin{document}
\title{Evaluating BLAST Results}
\subtitle{Assignment 7}
%
% You need the command \numberofauthors to handle the 'placement
% and alignment' of the authors beneath the title.
%
% For aesthetic reasons, we recommend 'three authors at a time'
% i.e. three 'name/affiliation blocks' be placed beneath the title.
%
% NOTE: You are NOT restricted in how many 'rows' of
% "name/affiliations" may appear. We just ask that you restrict
% the number of 'columns' to three.
%
% Because of the available 'opening page real-estate'
% we ask you to refrain from putting more than six authors
% (two rows with three columns) beneath the article title.
% More than six makes the first-page appear very cluttered indeed.
%
% Use the \alignauthor commands to handle the names
% and affiliations for an 'aesthetic maximum' of six authors.
% Add names, affiliations, addresses for
% the seventh etc. author(s) as the argument for the
% \additionalauthors command.
% These 'additional authors' will be output/set for you
% without further effort on your part as the last section in
% the body of your article BEFORE References or any Appendices.

%\numberofauthors{3} %  in this sample file, there are a *total*
% of EIGHT authors. SIX appear on the 'first-page' (for formatting
% reasons) and the remaining two appear in the \additionalauthors section.
%
\numberofauthors{1}
\author{
	\alignauthor Caitlin Ross\\
	\affaddr{Computer Science Department, Rensselaer Polytechnic Institute} \\
	\email{rossc3@rpi.edu}
}
% There's nothing stopping you putting the seventh, eighth, etc.
% author on the opening page (as the 'third row') but we ask,
% for aesthetic reasons that you place these 'additional authors'
% in the \additional authors block, viz.

\date{30 July 1999}
% Just remember to make sure that the TOTAL number of authors
% is the number that will appear on the first page PLUS the
% number that will appear in the \additionalauthors section.

\maketitle

\begin{abstract}

\end{abstract}


\section{Problem Statement}
I was given a fasta file with a DNA sequence.  The problem here is to run this sequence through BLAST, report the top 10 results, and do some research on the species the sequence belongs to.


\section{Methods and Results}
As my Ross.fasta file contained a DNA sequence, I used nucleotide BLAST to run on the sequence. The top 10 results from BLAST are shown:

\begin{enumerate}
 \item Plasmodium falciparum 3D7 casein kinase 1, PfCK1 (PfCK1) mRNA, complete cds \\
 match length: 1207 \\
 e-value: 0.0 \\
 bit-score: 828 \\
 \% identity: 100\% \\
 strand: Plus/Plus \\
 
 \item Plasmodium falciparum casein kinase 1 (CK1) mRNA, complete cds \\
  match length:1297  \\
 e-value: 0.0 \\
 bit-score: 817 \\
 \% identity: 99\%\\
 strand: Plus/Plus \\
 
 \item Plasmodium reichenowi casein kinase 1 (CK1), partial mRNA \\
  match length: 972 \\
 e-value: 0.0 \\
 bit-score: 767 \\
 \% identity: 99\% \\
 strand: Plus/Plus \\
 
 \item Plasmodium berghei strain ANKA casein kinase 1 (PB000530.02.0), partial mRNA \\
  match length: 735 \\
 e-value: 1e-149 \\
 bit-score: 540 \\
 \% identity: 89\% \\
 strand: Plus/Plus \\
 
 \item Plasmodium yoelii yoelii str. 17XNL casein kinase i (PY06011) partial mRNA \\
  match length: 972 \\
 e-value: 5e-148 \\
 bit-score: 534 \\
 \% identity: 89\% \\
 strand: Plus/Plus \\
 
 \item Plasmodium vinckei vinckei CK1/CK1 protein kinase partial mRNA \\
  match length: 972 \\
 e-value: 5e-143 \\
 bit-score: 518 \\
 \% identity: 89\%\\
 strand: Plus/Plus \\
 
 \item Plasmodium berghei strain ANKA hypothetical protein (PB300304.00.0) partial mRNA \\ 
  match length: 549 \\
 e-value: 9e-81 \\
 bit-score: 311 \\
 \% identity: 90\% \\
 strand: Plus/Plus \\
 
 \item Plasmodium chabaudi chabaudi casein kinase 1 (PC000102.03.0) partial mRNA \\
  match length: 540 \\
 e-value: 1e-79 \\
 bit-score: 307 \\
 \% identity: 91\% \\
 strand: Plus/Plus \\
 
 \item Plasmodium falciparum 3D7 chromosome 11, complete sequence \\
  match length: 2038337 \\
 e-value: 9e-71 \\
 bit-score: 278 \\
 \% identity: 99\% \\
 strand: Plus/Minus \\
 
 \item Plasmodium yoelii genome assembly PY17X01, chromosome : 9 \\
  match length: 1824586 \\
 e-value: 2e-22 \\
 bit-score: 117 \\
 \% identity: 95\% \\
 strand: Plus/Plus \\
 
\end{enumerate}

The highest match shows that the sequence is the CDS for casein kinase for the Protozoan (unicellular eukaryote) Plasmodium falciparum.  Plasmodium falciparum is a parasite that can cause malaria.  
The casein kinase is an enzyme that regulates signal transduction pathways in eukaryotes.  

a. The sequence identity
b. The genomic type (gene, CDS, exon, intron etc.)
c. The function of the sequence. If it is a segment of a larger structure such as a gene,
report the gene’s function.
d. The species the sequence was derived from. For the species, report the species name,
the common name, the species environment and behavior, the species taxonomy, i.e.
the species genus, species etc. 

%
% The following two commands are all you need in the
% initial runs of your .tex file to
% produce the bibliography for the citations in your paper.
\bibliographystyle{abbrv}
%\bibliography{sigproc}  % sigproc.bib is the name of the Bibliography in this case

% You must have a proper ".bib" file
%  and remember to run:
% latex bibtex latex latex
% to resolve all references
%
% ACM needs 'a single self-contained file'!
%
%APPENDICES are optional
%\balancecolumns

\balancecolumns
% That's all folks!
\end{document}
